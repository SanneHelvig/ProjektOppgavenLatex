The drag of a small-scale rotating wind turbine model was measured in a wind tunnel, and compared to the drag of different actuator disks, in order to find the actuator disk that most resembles the rotating model. The actuator disks were created with two different designs, one with uniform holes and one non-uniform, and three different solidities were used for each design. A solid disk was also tested as a reference. The drag on the actuator disks was measured in the wind tunnel in the same manner as with the rotating model, using five different Reynolds numbers. When conducting the experiments, several measurements of the rotating models had to be conducted due to deviations in the measurement data. Some of these deviations were regarded as outliers and discarded, and the average was taken of the remaining measurements. This resulted in a drag coefficient profile which was compared to the drag coefficient profiles representing the different actuator disks. The disk with a non-uniform design and a solidity of 35\% turned out to be the closest match, while the disk with uniform holes and a solidity of 35\% was the second closest match. Assuming a Gaussian distribution of the drag coefficients, both of the disks with 35\% solidity overlap with the rotating model's drag coefficient within their standard deviations, showing that it is reasonable to use these actuator disks in place of a rotating model. 


%Assuming that the drag coefficient from the rotating model is correct and assuming a Gaussian distribution of the drag coefficients from the actuator disks, both of the disks with 35\% solidity had the rotating models drag coefficient within its standard deviation, showing that it is reasonable to use these actuator disks to mimic the rotating model. 


%and the drag coefficients resulting from all the disks were compared to the drag coefficient of the rotating model. 






%In this thesis, the drag of a small-scale rotating wind turbine model is measured in a wind tunnel. Actuator disks, with three different solidities and two different layouts, were designed and 3D printed. Then, the drag on the actuator disks was measured, and the results were compared to the rotating wind turbine models in order to find the best match. The disk with a non-uniform design and a solidity of 35\% turned out to be the closest match, while the disk with a uniform holes design and a solidity of 35\% was the second closest match.  

%should be in norwegian too 


%less than 200-300 words, shorter than conclusion , con should cover every important finding 

%can have citations in conclusion but not abstract %avoid acronyms in both 

%transducer and human error 
%reviewpapers!! %antonia pp compare, see dynamic differences 
%disks are based on this... 

%picture of PIV setup 
%in conclu, some problems, I learned did, created these curves, and these are the closest ones to go fort with 