Motstandskraften til en småskala roterende vindturbinmodell ble målt i en vindtunell, og sammenlignet med motstandskraften til ulike aktuatordisker, for å finne den aktuatordisken som ligner mest på den roterende modellen. Aktuatordiskene ble laget med to ulike design, en med uniforme hull og en ikke-uniform, og tre ulike tettheter ble brukt for hvert design. En solid disk ble også testet som en referanse. Motstandskraften til aktuatordiskene ble målt i vindtunellen på samme måte som for de roterende modellene, ved å bruke fem ulike Reynoldstall. Flere målinger av de roterende modellene ble gjennomført på grunn av avvik i måledataene. Noen av disse avvikene ble ansett som uteliggere og forkastet, før gjennomsnittet av de gjenværende målingene ble funnet. Dette resulterte i en dragkoeffisientprofil som ble sammenlignet med dragkoeffisientprofilene til de ulike aktuatordiskene. Disken med et ikke-uniformt design og 35\% tetthet viste seg å samstemme best, etterfulgt av disken med uniforme hull og 35\% tetthet. Hvis man antar at dragkoeffisientene er normalfordelte, overlapper begge diskene med 35\% tetthet med den roterende modellen sin dragkoeffisient innenfor sine standardavvik, noe som viser at det er rimelig å bruke disse aktuatordiskene i stedet for en roterende modell.    


%The drag of a small-scale rotating wind turbine model was measured in a wind tunnel, and compared to the drag of different actuator disks, in order to find the actuator disk that most resembles the rotating model. The actuator disks were created with two different designs, one with uniform holes and one non-uniform, and three different solidities were used for each design. A solid disk was also tested as a reference. The drag on the actuator disks was measured in the wind tunnel in the same manner as with the rotating model, using five different Reynolds numbers. When conducting the experiments, several measurements of the rotating models had to be conducted due to deviations in the measurement data. Some of these deviations were regarded as outliers and discarded, and the average was taken of the remaining measurements. This resulted in a drag coefficient profile which was compared to the drag coefficient profiles representing the different actuator disks. The disk with a non-uniform design and a solidity of 35\% turned out to be the closest match, while the disk with uniform holes and a solidity of 35\% was the second closest match. Assuming a Gaussian distribution of the drag coefficients, both of the disks with 35\% solidity overlap with the rotating model's drag coefficient within their standard deviations, showing that it is reasonable to use these actuator disks in place of a rotating model. 