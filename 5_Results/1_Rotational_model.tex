%After conducting the measurements, profiles showing the drag and the drag coefficient as a function of Reynolds number could be created for each set of \gls{AD}s, as well as for the \gls{RWTM}s. 

It turned out that several measurement sets were needed to be conducted for the rotating models, due to several deviations in the data. In this section, the measurement data and the process of treating it will be presented, together with the final average drag coefficient of the rotating models. After, the drag and drag coefficient of all the different \gls{AD}s will be presented and compared to the average results for rotating models.  

%In this section, the drag and drag coefficient profiles are presented and discussed, 

%as well as the calculated average Cd and the standard deviations. In the end, the measurement noise is discussed. 

\section{Rotational model}

The first measurement set conducted using three \gls{RWTM}s resulted in a drag coefficient that seemed relatively independent of Reynolds number for four of the measured wind velocities, but with a noticeable deviation at 5 \si{\m\per\s}. To investigate whether this deviation was due to a measurement error, a second measurement set was conducted, this time using three new \gls{RWTM}s. This second measurement gave more of an expected result at 5 \si{\m\per\s}, however showed a deviation at 15 \si{\m\per\s}. Due to continuous deviations, however differing in size and appearing at different velocities, six measurements were eventually conducted. They were all done with different sets of \gls{RWTM}s, except for measurement set three and four, which were done using the same set of models. The resulting drag coefficients can be seen as a function of Re in figure \ref{fig:RotationalCD}.

%Thus, a third measurement, once again with three new rotating WT models, was conducted. Finally, a fourth measurement was carried out, this time using the same models as during the third measurement. 

\begin{figure}
    \centering
    \includegraphics[width=\linewidth]{0_Images/RotationalCDRe.eps}
    \caption{The drag coefficient for the rotating models, obtained through six rounds of measurements.}
    \label{fig:RotationalCD}
\end{figure}

As can be seen, there is some variation between the different measurements. The drag coefficients resulting from the third and the fourth measurement set, conducted using the same models, are quite similar at 5 \si{\m\per\s}, 7.5 \si{\m\per\s} and 12 \si{\m\per\s}, and at 15 \si{\m\per\s}, they completely overlap. This seems to show that the measurement is to some degree repeatable, and that one of the reasons for the varying results is simply that there are small differences between the \gls{RWTM}s. These differences can for example be related to the friction between the rotating blades and the hub that they rotate around. In addition, the blades stay onto the hub due to a small piece of see-through plastic, that differs in size between the models. If it is too big, it might fasten the blades too tightly, causing added friction. If it is too small, the blades might be too loose, and they may start to oscillate. These types of differences were evident during the measurement phase, as several times the measurements had to be stopped midway due to one of the turbines suddenly not rotating anymore, and once because the rotating blades fell of the model.  
%how well they are connected. Be it too tight, there will be added friction.Be it too loose, the blades may start to oscillate


However, even between the third and fourth measurement set, there is a noticeable difference at 7.5 \si{\m\per\s}, showing that differences between the \gls{RWTM}s is not the only cause for the varying results. Other possible causes of this variation may be related to fluctuations in the applied wind velocity and to noise in the force plate and the electrical equipment used. Human error is also an important factor, as the models were placed in the wind tunnel by hand, and the turbine blades were not necessarily always exactly perpendicular to the incoming flow direction.   

To investigate the results further, the drag resulting from the different measurements were plotted as a function of Reynolds number, as seen in figure \ref{fig:RotationalDrag}


\begin{figure}
    \centering
    \includegraphics[width=\linewidth]{0_Images/RotationalDragRe.eps}
    \caption{The measured drag for the rotating models, obtained through six rounds of measurements.}
    \label{fig:RotationalDrag}
\end{figure}

The drag at 15 \si{\m\per\s} is significantly lower than the drag at 12.5 \si{\m\per\s} for the second measurement set. The same is the case for the third measurement set, where the drag at 12.5 \si{\m\per\s} is significantly lower than the drag at 10 \si{\m\per\s}. This is not physical, and thus it is assumed that these two measurements are errors. 

The measurements at 12.5 \si{\m\per\s} were studied further. The average drag of the measurements that seem to cluster is 0.091 \si{\newton}, with a standard deviation of 0.0114 \si{\newton}. Looking at the measurement value from the fourth measurement set, it has a drag of 0.0387 \si{\newton}, and thus it is over four standard deviations away from the mentioned average. Hence, this value is considered to be an outlier. (It is worth noting that this outlier coincides with the already discarded drag measurement from the third measurement set. The wind tunnel seemed to produce considerably large amounts of noise on the signal for velocities between 11 and 13 \si{\m\per\s} for these two measurement sets that were conducted with the same set of rotating models, which might explain this repeated deviation.)
%The drag measured during the fourth measurement coincides with the disregarded drag from the third measurement, and is thus also regarded as an outlier. The wind tunnel did, for some unknown reason, seem to produce larger amounts of noise on the signal for velocities between 11 and 13 m/s for these specific measurement sets, which might explain this repeated deviation. 

A similar study was conducted for 10 \si{\m\per\s}. Five of the measurements seem to coincide, with an average drag of 0.0524 \si{\newton} and a standard deviation of 0.0047 \si{\newton}. The drag resulting from the fourth measurement set is 0.0332 \si{\newton}, and thus more than four standard deviations away from the average. Hence, this value is also considered to be an outlier.  

%The drag at 10 m/s for the fourth measurement is higher than the drag at 7.5 m/s, however the value is lower than for all the measurements which seem to coincide quite well, and thus this value is considered as an outlier. The initial suspicious value, measured at 5 m/s in the first measurement, results in a $C_d$ larger than one, which seems unlikely, and hence this value is also regarded as an outlier. 
%AGREE? 

In total, four measured drags has been discarded, and thus the four associated drag coefficients were removed. In order to achieve a representative value for the drag coefficient of the \gls{RWTM}s, the average of the remaining drag coefficients was taken at each velocity. This resulted in the drag coefficients seen in figure \ref{fig:RotationalAvg}. 
%It is assumed that this average is representative for the average one would have gotten if all the rotating models been tested and all the tests had been conducted multiple times. 

%The outlier seem to be spread out both in terms of velocity at which they occur and in terms of whether they exceed or fall below the other values. One can argue that if taking a large number of new measurements, they would have a Gaussian distribution about a mean, and that taking the average of the values that seemingly coincide would be representative for this total mean. %Should probably be extended 



\begin{figure}[h!]
    \centering
    \includegraphics[width=\linewidth]{0_Images/RotationalAverageRe.eps}
    \caption{The average drag coefficient for the rotational models at each wind velocity, based on the six conducted measurements, after removing the assumed errors and outliers.}
    \label{fig:RotationalAvg}
\end{figure}


%I BELIEVE THIS COMES LATER 
%Assuming that $C_d$ is Reynolds number independent for such a short span at Reynolds numbers, the average over these measurement points is taken, resulting in an average $C_d$ of 0.585. Based on all the applied values, the standard deviation at hand is... Thus, when creating the ADs, this is the desired drag coefficient. 

\FloatBarrier