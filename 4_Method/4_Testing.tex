\section{Testing}
The rig was connected to the force plate using bolts, in such a way that the aluminium cylinder came up through the hole on the bottom of the tunnel. The rest of the hole was covered with thick tape. Careful consideration was taken when adding the tape, so that the aluminium cylinder did not touch anything, as that would affect the force measurements. 

Given that the size of the wind turbine models is quite small, the turbines were tested in the wind tunnel three at a time, to ensure that the drag would be of an order that the instruments were able to measure and of an order were slight changes in the design resulting in slight changes in drag would be noticeable. The turbines were placed with a distance of 4D between them along the steel bar. The same was the case for the actuator disks. %KILDE  

Even though the turbine models were magnetic, it proved problematic to make them stay in the same position, with the turbine perpendicular to the wind direction, as the wind velocity increased. Thus, the models were connected to the steel bar using small pieces of tape. The 3D printed towers were able to stay in the right position on the base by themselves, still they were taped to the base like the rotating turbine models, to make sure the cases were comparable. 

As the drag is the force of interest in this work, only the force in the x-direction is measured. The force was measured for five different wind velocities; ~5 m/s, ~7.5 m/s, ~10 m/s, ~12.5 m/s and ~15 m/s. This corresponds to Reynolds numbers all of the order $10^4$. Since the velocity was changed by manually turning a wheel, a slight difference in the velocities occured between the different measurements. 

The force plate drifted over time, as is often the case with force measuring equipment. To take this into consideration when measuring the forces, zero measurements were conducted before and after every measurement. A 20 second tare measurement was first conducted. The wind tunnel was then turned on, with the velocity initially set to about 1 m/s, and then turned up to the wanted value. A measurement lasting one minute was then conducted. The velocity was once again reduced to about 1 m/s, and the wind tunnel was shut off. After the wind tunnel had quiet down and there was close to no moving air inside, another 20 second tare measurement was conducted. When measuring, a sampling rate of 1000 samples per second was used. 

Besides measuring the drag on the rotating turbine models and on all the different sets of actuator disks, a measurement was also conducted measuring only the drag on the base and the towers, without having any disks connected to them. 

Define spacial axes! 

%Based on the resulting drag profiles, the designs and solidities were to be adjusted. The iterative prosedure goes on until an actuator disk that has the same drag as the wind turbine model is presented.  
