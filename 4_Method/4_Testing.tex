\section{Testing}
Given that the size of the \gls{WTM}s was quite small, the models were tested in the wind tunnel three at a time, to ensure that the drag would be of an order that the instruments were able to measure and of an order where slight changes in the design resulting in slight changes in drag would be noticeable. The \gls{WTM}s were placed with a distance of 4D between them along the steel bar.  %KILDE  

Even though the \gls{RWTM} were magnetic, it proved problematic to make them stay in the same position, with the turbine perpendicular to the incoming flow direction, especially as the wind velocity increased. Thus, the rotating models were connected to the steel bar using small pieces of tape. The 3D printed towers were able to stay in the right position on the base by themselves. Still, they were taped to the base like the \gls{RWTM}s, to make sure the cases were comparable. 

As the drag is the only force of interest in this work, only the force in the x-direction is measured. The force was measured for five different wind velocities; 5 \si{\m\per\s}, 7.5 \si{\m\per\s}, 10 \si{\m\per\s}, 12.5 \si{\m\per\s} and 15 \si{\m\per\s}. This corresponds to Reynolds numbers 14 860, 22289, 29 719, 37149 and 44 579, respectively, so all of the order $10^4$. When calculating Reynolds number, the characteristic length used is the diameter of the \gls{AD}s, 0.045 \si{\m}. Since the velocity was changed by manually turning a wheel, a slight difference in the velocities occurred between the different measurement sets. 

The force plate drifted over time, as is often the case with force measuring equipment. To take this into consideration when measuring the forces, zero measurements were conducted before and after every measurement. A 20 \si{\s} tare measurement was first conducted. The wind tunnel was then turned on, with the velocity initially set to about 1 \si{\m\per\s}, and then turned up to the desired value. A measurement lasting 60 \si{s} was then conducted. The velocity was once again reduced to about 1 \si{\m\per\s}, and the wind tunnel was shut off. After the wind tunnel had quiet down and there was close to no moving air inside, another 20 \si{\s} tare measurement was conducted. The wind tunnel needed about 10 \si{\minute} before one could be sure that the air inside was still, meaning that the measurements were quite time consuming. When measuring, a sampling rate of 1000 \si{\hertz} was used.  

Besides measuring the drag on the \gls{RWTM}s and on all the different sets of \gls{AD}s, a measurement set was also conducted having only on three towers on the base, without having any disks connected to them. Thus, the drag of the base and the towers was quantified. 

%measuring only the drag
