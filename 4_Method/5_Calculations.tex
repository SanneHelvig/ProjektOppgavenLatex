\section{Calculations}

For each disk at each wind velocity, the data collected from the wind tunnel consisted of a time series of voltages corresponding to the measured force and the measured wind velocity, as well as the times for when the first zero measurement and the second zero measurement was conducted and when the 60 second measurement started. Using Matlab, this data was treated.

A linear drift of the force plate was assumed. Thus, using the two zero measurement values, a linear function approximating the drift was created. The part of this linear function corresponding to the 60 seconds where the force measurement was conducted, was extracted. For each measured force in the time series, the corresponding drift was subtracted. After, the average force over the time series was calculated, as well as the variance and standard deviation. 

In the same matter, the drift was subtracted and the average force was calculated for the measurements of only the base and the tower. This was done for each of the different wind velocities. These averages were then subtracted from the averages calculated earlier, so that the final drag force was the force only on the disks, excluding the towers and the base. 

Finally, this calculated drag was divided by three, so as to only consider the drag on one disk. This force was used in calculating the drag coefficient, as well as the total swiping area of the rotating turbine model, being $\pi r^2$. 

Another value collected from the measurements was the average temperature during the 60 seconds of measuring, used to decide on the appropriate value for the air density when calculating the drag coefficient. 

Removing the towers: 
\cite{Aubrun2019} concluded that the discrepancy in rod fixation and distance between wall and disc center can generate different wake downwasg that might explain differences. 