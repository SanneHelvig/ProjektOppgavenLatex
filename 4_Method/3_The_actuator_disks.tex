\section{The actuator disks}
As mentioned, there are no standard way of designing actuator disks. It was desirable to create ADs with multiple solidities, and a possible method that suggested was to create two ADs that are connected and can be rotated relative to each other, in order to change the solidity. However, this seemed hard to achieve at such small scales as are relevant in this work. In addition, the author was skeptic to the method based on the results of \cite{Pierella2010}, showing that a monoplane and a biplane AD made with the same shape and porosity produced different drags, making the cases not comparable. Since making a monoplane disk is perhaps simpler and easier to recreate, this method was preferred. 
 

\subsection{Computer-aided design and 3D printing}
The actuator disks, as well as their stands, were designed using SolidWorks. Cura was used to turn the designs into readable code for the 3D printers, and the parts were then printed using a printer of the type Ultimaker 2+. The material used was PLA.
%3D printing was chosen as it is  repeatable 

A significant limitation occurred during the design process. The 3D printers available could not print thinner than 0.4 mm, meaning that each line in the disks had to be at least 0.4 mm. However, printing lines of 0.4 mm proved troublesome, and it was decided that all lines should be equal to or thicker than 0.5 mm. This is a significant size given that the disks are in themselves of such small dimensions. So it turned out to be a limit to how porous the disks can be made. 

In general, a concern when using 3D printers is the fact that the print is not a hundred percent equal to the design - small variations may occur, making the actuator disks with the same design slightly different from each other. These slight variations matter more when the overall size of the design is so small compared to a larger design, however, the variations are still so few and small that they are considered to have close to no effect on the solidity or in making the disks differ from each other. In comparison, other methods of making ADs may also lead to minor differences. 

\subsection{Design of the tower}
The tower was designed to have the exact same dimensions as the given wind turbine model's tower. Most importantly, it had a hub height of 65 mm. Underneath there was made a hole that could fit a cylindrical neodymium magnet with a diameter of 10 mm, a height of 2.5 mm and a strength of 0.9 kg. The actuator disks were made to be interchangeable, and thus the end of the tower where the actuator disks will be connected is made slightly thinner in order to fit into the designated holes in the actuator disks. Three towers were printed. 

\subsection{Actuator disk design}
The actuator disks were designed with a diameter of 45 mm, to match the wind turbine models. The thickness of the disks is 2.5 mm. 

Two different designs of actuator disks have been tested. The first has numerous equally-sized holes spread symmetrically around the center point of the disk. This design is meant to be similar to those actuator disk designs made by a thin metal grid, comparable to a grid turbulence generator. The second design is also symmetric around the center point, but this one has rectangular (however filled in to avoid sharp corners) holes that vary in size with radial distance, increasing in size as the radial coordinate increases. Thus, the solidity decreases with radial coordinate, matching the characteristics of an actual WT.  

For each of these configurations, two degrees of solidity was used as an initial try. The chosen values were 60\% and 40\%. A solid disk was also made and tested as a reference case. Three disks of each design and solidity were printed. 

Each disk was as mentioned made with a small hole in the center, used to connect the stand to the disks. Thus, this hole was filled in during the tests in the wind tunnel, and did not affect the solidity. This resulted in a larger solidity in the center of the disks, which can be argued to represent the nacelle of a WT.  

Based on the resulting drag profiles from the initial round of testing, two sets of actuator disks with a solidity of 35\% were designed and made. The first was made based on the design with equally sized holes. Due to the mentioned limitations regarding the printing thickness, there was a limit to how low the solidity could get, and providing a solidity less than 39\% proved problematic. Hence the design was slightly changed, allowing for the holes to also cover the edges of the actuator disk. It was kept in mind that this results in a different disk circumference, and that this might result in a drag force unrelated to the previously tested disks. The second design was made as the designs with rectangular holes that vary in size with radial distance.  

%PICTURE OF DISKS AND TOWER 