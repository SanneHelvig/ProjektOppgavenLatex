\section{Experimental setup}
%INCLUDE
%calculate reynold numebr 

In order to measure the drag on the wind turbine models and the actuator disks, a wind tunnel and a force plate is needed as part of the experimental setup. There was also the need to construct a rig on which the turbines could be placed inside the wind tunnel. 

%Hvorfor og hva vi skal gjøre 

\subsection{Force measurer, wind tunnel and associated equipment}
The wind tunnel used is one meter wide and a half meter tall. It has a maximum velocity of 35 m/s and a turbulence intensity of ... One can get inside the wind tunnel by removing a glass window which is ... m wide and ... m tall. The wind velocity can be changed by manually turning a wheel that changes the position of the valves in the tunnel. In the bottom plate of the tunnel there is a small hole, making it possible to connect the item one is measuring forces on to the load cell underneath the tunnel.  

Underneath the wind tunnel is a force plate of the type AMTI BP400600HF 1000, able to measure the force and moments components along the x-, y- and z-axes. It can measure forces as low as ... and has an accuracy of ...  

The voltage signal from the load cell is sent through an amplifier. Afterwards, it is sent through a low pass filter, with a cut-off frequency of 1000 Hz. The data was treated using LabView. 

Inside the tunnel there is a sensor measuring the temperature, and a pitot tube measuring the pressure. 

A potential uncertainty related to the wind tunnel is the fact that the pitot tube, which measures the pressure and is used to quantify the wind velocity, is placed in the centre of the tunnel. Thus, the velocity measured is not necessarily the same as the velocity that hits the actuator disks close to the ground of the tunnel. Due to wall boundary layers, the velocity hitting the disks is likely lower than the measured and registered velocity. 


\subsection{The rig} 
The test rig consisted of a magnetic steel bar of 0.5 m stretching across the width of the wind tunnel, on top of a aluminium cylinder which connects the bar to an aluminium plate, that in turn can be strapped to the load cell underneath the wind tunnel. The bar was lifted about a centimeter above the ground floor of the tunnel. The measurements can be seen in the figure. 
%Figur 

Initially, it was desired to have the steel bar be almost as long as the width of the wind tunnel, in order to avoid affecting the flow outside of what is already the boundary layer in the tunnel. Similarly, it was desired to have the hub of the turbines exactly in the vertical middle of the tunnel, to avoid the boundary layers. However, this was not doable, as the hole in the bottom of the wind tunnel was limited in size, which meant that the steel bar could only be connected to the load cell underneath the tunnel through one aluminium cylinder with a small diameter of about 2 cm. The length of the metal bar had to bee shortened in order to avoid bending and flapping. 







 