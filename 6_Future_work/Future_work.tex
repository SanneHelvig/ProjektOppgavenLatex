As concluded in \cite{}, in order to create an AD matching a WTM, the wake must be similar. Matching the drag coefficient is only the first step in this process. 

Going forth, the rotating disks and the ADs with the closest matching drag coefficient will be studied using Particle Image Velocimetry, both in front of and behind the models. This method can be used to, amongst others, study the velocity deficit and the turbulence intensity in the wake. It is also desired to integrate the wake in order to find the axial induction factor. 

Further, the plan is to acquire 100-150 rotating WTMs, and set them up as a wind farm in a larger wind tunnel. The same thing will be done with the AD that most closely matched the drag and wake. This will be done in order to study the effect and suitability of using ADs when modelling large wind farms. 


%A natural extension to this work would be to create more actuator disks, with solidities , to see if it is possible to match the drag coefficient profile of the rotating turbine models even better.  

%What I will do in my masters 

%Future plans and outlook
%Describe what you will do moving forward with this data for next term.  Maybe even propose a matrix of tests you intend to look at or write a little bit about the problems in the area you intend to investigate and how you will solve them.