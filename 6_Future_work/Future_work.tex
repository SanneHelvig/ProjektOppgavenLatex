As concluded in \cite{}, in order to create an \gls{AD} matching a \gls{RWTM}, the wake must be similar. Matching the drag coefficient is only the first step in this process. 

Going forth, the rotating models and the two \gls{AD}s with the closest matching drag coefficient will be studied using Particle Image Velocimetry. Thus, both the flow in front of the models and the wake behind the models can be studies. This method can be used to examine the velocity deficit and the turbulence intensity in the wake. It is also desireable to integrate the wake in order to find and compare the axial induction factor. The problem with using mono-plane actuator disks is the lack of a way to add a rotational momentum to the flow, and the effect of this will be studied. 

Further, the plan is to acquire 100-150 \gls{RWTM}s, and set them up as a wind farm in a larger wind tunnel. Equally many \gls{AD}s will be made, using the design that most closely match the drag and the wake, and they will also be set up as to simulate an entire wind farm. This will be done in order to study the similarities and differences between the models when used for wind farm modelling, and to some extent determine the suitability of using \gls{AD}s when modelling large wind farms. 
%more effect and


%A natural extension to this work would be to create more actuator disks, with solidities , to see if it is possible to match the drag coefficient profile of the rotating turbine models even better.  

%What I will do in my masters 

%Future plans and outlook
%Describe what you will do moving forward with this data for next term.  Maybe even propose a matrix of tests you intend to look at or write a little bit about the problems in the area you intend to investigate and how you will solve them.